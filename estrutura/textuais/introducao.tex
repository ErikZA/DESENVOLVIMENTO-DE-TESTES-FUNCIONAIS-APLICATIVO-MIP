\chapter{INTRODUÇÃO}
\label{chap:introducao}


% Desenvolver softwares com qualidade tem sido um desafio da era da computação, contudo a engenharia de software vem aprimorando a organização e qualidade de um software, dividindo a sua produção em fases bem definidas, padrões e metodologias entre outros modelos de desenvolvimento que apoiam a construção de um software. Uma metodologia de trabalho bem desenvolvida possui um conjunto de técnicas que visão a construção de um software com qualidade e de forma eficaz. Uma metodologia de teste deve estar presente no processo de desenvolvimento para o aumento  da qualidade e diminuir erros antes que o produto final chegue aos usuários. 

%  O teste de software como processo proporciona a última etapa na qual pode se realizara validação e a verificação das funcionalidades estabelecidas pelos requisitos dos usuários. \cite{PRESMA2016}. Basicamente o teste de software visa garantir a qualidade do software, diminuindo as imprecisões nos requisitos\footnote{Condição necessária para atingir certo objetivo.} e na prototipação\footnote{ Processo que tem como objetivo facilitar o entendimento dos requisitos, apresentar conceitos e funcionalidades do software}, auxiliando a equipe de desenvolvimento no entendimento das funcionalidades, proporcionando um bom funcionamento do produto. Os testes podem ser organizados em unidades chamadas de casos de teste, cada caso de teste é responsável por testar um requisito especifico. Um caso de teste contem passo a passo para a execução do teste e o resultado esperado após a execução dos passos definidos.

O teste de software permite a validação e a verificação das funcionalidades estabelecidas pelos requisitos dos usuários \cite{PRESMA2016}. Além disso, caso o código seja alterado, os testes permitem avaliar se as novas funcionalidades se integram sem problemas com as funcionalidades anteriores. O teste de software visa garantir a qualidade do software, diminuindo as imprecisões nos requisitos\footnote{Condição necessária para atingir certo objetivo.} e na prototipação\footnote{ Processo que tem como objetivo facilitar o entendimento dos requisitos, apresentar conceitos e funcionalidades do software}, auxiliando a equipe de desenvolvimento no entendimento das funcionalidades, proporcionando um bom funcionamento do produto. Os testes podem ser organizados em unidades chamadas de casos de teste, no qual cada caso de teste é responsável por testar um requisito específico. Um caso de teste contém passo a passo para a execução do teste e o resultado esperado após a execução dos passos definidos \cite{Hushalini}.

Teste funcional é uma técnica aplicada aos requisitos do sistema para verificar se o software faz o que foi proposto. Esta estratégia é utilizada geralmente para softwares que já foram implantados ou já estão em final de linha de produção. Outra técnica aplicada a unidades do software e componentes do sistema a fim de avaliar o comportamento interno do sistema é o teste de caixa-branca, onde trechos de código são testados para avaliar fluxos de dados, caminhos lógicos, condições de entrada e ciclos de repetição. Os testes em unidades se focam em caminhos específicos da estrutura de controle, e visam garantir máxima cobertura e detecção de erros, enquanto teste de integração de componentes têm foco nas entradas e saídas de dados durante a comunicação de diferentes pacotes do software \cite{PRESMA2016}.  

% No entanto, a atividade de teste é uma atividade demorada e custosa, e por conta deste motivo automatizar os testes é uma boa politica dentro de um ciclo de desenvolvimento de um software pois pode evitar falhas dos próprios testadores. Além disso caso alterações sejam solicitadas, os testes automatizados podem ser executados novamente para realizar a avaliação do estado da aplicação e se a funcionalidade está de acordo com suas especificações \cite{Hushalini}.

Portanto, a atividade de teste é uma atividade demorada e custosa. Por isso, frequentemente as atividades de teste podem ser negligenciadas devido à um prazo curto para a entrega do software proposto. Este é o caso do aplicativo para coleta e análise de dados do manejo integrado de pragas. O software foi desenvolvido como fruto de trabalhos de conclusão de curso da UTFPR-CP para atender uma demanda do EMATER - instituição paranense de extensão rural. A falta de teste dificulta a evolução do sistema, pois a adição de novas funcionalidades podem gerar erros em funcionalidades já existentes sem que isso seja percebido pelo desenvolvedor em tempo de desenvolvimento. 

\section{JUSTIFICATIVA}

Os resultados deste trabalho têm uma contribuição importante para a manutenção da qualidade e evolução do sistema MIP, reduzindo o impacto causado por alterações nos requisitos e no código fonte, possibilitando a verificação das funcionalidades através do processo de teste funcional. 

Mudanças durante o desenvolvimento de um software são inevitáveis, e podem afetar as funcionalidades já validadas causando um grande impacto no projeto. O teste de software é um processo que visa garantir o controle de qualidade de um sistema. O processo deve executar uma varredura completa na busca por defeitos, confirmando se o software se comporta como deveria mesmo em situações inesperadas, simulando o máximo de cenários possíveis, garantindo que todos os requisitos foram atendidos. 


Os testes funcionais são utilizados para verificar se a aplicação está apta a realizar as funções para as quais foi desenvolvida para fazer atendendo os requisitos do sistema. Estes testes depois de desenvolvidos podem ser utilizados diversas vezes, pois eles testam funcionalidades específicas de um sistema, portanto mesmo que o código seja modificado, novos módulos sejam integrados ou uma nova funcionalidade seja adicionada ao sistema é possível executar o mesmo teste, garantindo assim a integridade da funcionalidade validada anteriormente evitando a regressão do software.


Deste modo, o processo de teste visa garantir a entrega de um produto que pode ser submetido a novas modificações garantindo a integridade das funcionalidades já existentes e que possua qualidade. Qualidade é um ato de prevenção e só é atingida se o produto for construído de acordo com o processo bem definido. 


% Portanto este trabalho faz uma contribuição importante para a implantação do sistema MIP, que ainda não passou por uma fase de testes durante o seu desenvolvimento. O sistema foi desenvolvido com o proposito de atender a demanda de usuários do setor agrícola. Neste sentido, o software deve satisfazer a demanda dos requisitos funcionais definidos, os requisitos devem funcionar corretamente e estar de acordo com a especificação dos usuários. 

% O processo de teste aplicado ao MIP pode promovendo um ganho de qualidade e manutenibilidade detectando qualquer possível problema injetado durante o desenvolvimento.



\section{OBJETIVOS}

Nesta seção são apresentados o objetivo geral e os objetivos específicos a
serem alcançados no desenvolvimento da proposta.

\subsection{Objetivo Geral}

O objetivo deste trabalho é elaborar testes unitários em todas as unidades da aplicação e testes de integração para aumentar a qualidade e facilitar a evolução do aplicativo de coleta e análise de dados do manejo integrado de pragas (MIP). 


\subsection{Objetivos Específicos}

Para atingir o objetivo geral proposto, os objetivos específicos
apresentados a seguir foram definidos:



\begin{itemize}
\item Documentar os requisitos do sistema; 
 
\item Criar testes unitários com base nos requisitos do sistema;

\item Criar testes de integração para verificar a comunicação entre as diferentes unidades do sistema;

% \item Utilizar a metodologia de teste 3p x 3e para planejamento, preparação, especificação, execução e registro de resultados dos testes;

% \item Realizar a validação das funcionalidades por meio de testes de interface do usuário.
\end{itemize}

\section{ORGANIZAÇÃO DO TRABALHO}

O presente trabalho está dividido da seguinte maneira: O primeiro capítulo apresenta a introdução juntamente da proposta e os objetivos que devem ser atingidos, o segundo capítulo é composto pela fundamentação teórica onde são apresentados os conceitos dos processos e ferramentas que envolvem este trabalho, o terceiro capítulo apresenta de forma breve as ferramentas, tecnologia e metodologia de desenvolvimento que serão aplicados para atingir os objetivos, juntamente aos diagramas e requisitos que modelam o sistema, no capítulo seguinte é apresentado o cronograma para conclusão do projeto e no último capítulo as considerações finais.   
